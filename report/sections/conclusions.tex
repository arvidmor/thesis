In this paper I present an implementation of Dynamic Software Updates (DSU) utilising the relatively novel memory technology FRAM. The implementation is based on delta encoding, where the difference between the old and new binary image is stored in a patch file describing the changes necessary to update the application. The patch file is then applied to the binary image in two stages, Decode and Apply, updating it to the new version. The implementation is intended for use in embedded real-time systems, where resource constraints are a major concern. The implementation is evaluated on a Texas Instruments MSP430FR5969 microcontroller, and the results along with proof-of-concept updates show that the implementation is feasible. Using FRAM, I show that the DSU process can run orders of magnitude faster when compared to traditional storage mediums such as Flash. This allows the process to run in ever more constrained environments and allows for more flexibility in the updates that can be performed. The implementation is not without its drawbacks, however, and three main ideas for future work are presented. Firstly, the implementation is quite inefficient in its memory utilisation. Secondly, the decode phase of the DSU process is the most time consuming, and thus the most critical to optimise. Finally, the current implementation has no way of guaranteeing atomicity of the update process. Future work could focus on these areas as discussed further in section \ref{sec:future_work}. 
