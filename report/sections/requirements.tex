\subsubsection*{Problem statement}
As described in previous sections, DSU in embedded and real-time systems is not a new concept, but there are some key aspects that are yet to be considered in the research. 

Firstly, the currently existing work in the field of DSU have been focused on systems with more relaxed real-time requirements, where the overhead of the update process is not as critical. This project aims to investigate the use of DSU in systems with tighter real-time deadlines, where the overhead of the update process must be kept to a minimum.

To meet these deadlines, the utilisation of FRAM is proposed as a potential solution, and this research aims to investigate the performance impact that the technology has on the update process.  

\subsubsection*{Requirements}
As a way of containing the scope of the project, and to provide a clear goal for the research, the requirements of the work need to be clearly defined. There are a multitude of requirements to consider in the context of DSU, both functional and non-functional. Mlinarić describes in \cite{dsuChallenges} five key requirements for any DSU implementation, namely: \textit{Availability}, \textit{Correctness}, \textit{Flexibility}, \textit{Performance}, and \textit{Simplicity}. In addition to these general requirements for DSU, there are also requirements that are specific to the context of embedded systems, which may be considered subcategories of the above; \textit{Energy efficiency}, \textit{Memory usage}, and \textit{Security}.
\todo{Maybe expand on these}

The main focus of this research is the performance of the update process, however other aspects such as security and correctness are considered when evaluating the results, as they are factors which may impact the performance, and thus, the feasibility of the solution.