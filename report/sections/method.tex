\subsection{The Platform}
The platform used for the experiments in this project is the Texas Instruments MSP430FR5994 microcontroller. The MSP430FR5994 features a 16-bit RISC architecture and is equipped with 256KB of FRAM (Ferroelectric Random Access Memory), 8KB of SRAM, and 2KB of flash memory. The relatively large FRAM is particularly useful for this project since it is a non-volatile memory which at the same time provides high speed writes and low power consumption. This comes as a result of FRAM being writeable in a bit-wise manner, as opposed to Flash memory or EEPROM which requires entire pages to be erased before new data can be written, slowing down the operations \cite{framReport}. A side-effect of FRAM being non-volatile is that the binary image of the program is stored in the same place that it is executed from, and can be done in-place with little overhead. 

\subsection{The update process}
The update process is initiated by sending a command to the microcontroller over a serial connection. The command is received by the microcontroller and triggers the update process. The update process is divided into three main steps: patch file generation [\ref{sec:patchfile}], updating, and the finalization. The following sections will describe each step in detail.

\subsubsection{Patch file generation}\label{sec:patchfile}
Another important part of the update process is the generation of the patch file. The patch file must be minimal in size to keep the transfer- and parse times within the expected idle time. This naturally excludes simply transferring the entire binary image as an option. Instead, a \textit{diff}-file is generated by a server system by comparing the current binary to the patched one. For sufficiently small updates, the diff file could be as simple as stating addresses to changed instructions and/or data, and the new values for said addresses. This is however not ideal in the case of bigger updates since changes could cause portions of the binary to be shifted, which can be expressed with a \textbf{Shift} operation instead of individual byte- or word-level changes. Thus, a better approach is to use a binary diffing tool such as \textit{BinDiff} to generate a patch file.

\todo{Describe problem of finding instruction addresses}


