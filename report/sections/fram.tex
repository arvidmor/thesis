This section describes in further detail FRAM, the memory technology which is used as the storage medium for the update process in this research. 

The platform used for the experiments in this project is the Texas Instruments MSP430FR5994 16MHz microcontroller. The MSP430FR5994 features a 16-bit RISC architecture and is equipped with 256KB of FRAM (Ferroelectric Random Access Memory), 8KB of SRAM, and 2KB of flash memory. The relatively large FRAM is particularly useful for this project since it is a non-volatile memory which at the same time provides high speed writes and low power consumption. This comes as a result of FRAM being addressable in a bit-wise manner, as opposed to Flash memory or EEPROM which requires entire pages to be erased before new data can be written \cite{framReport}. 

A side-effect of FRAM being non-volatile is that the binary image of the program is stored in the same place that it is executed from, and can thus be updated in-place. This is a key feature that makes FRAM a suitable candidate for DSU in embedded systems. Consider the alternative, a platform using Flash memory; One approach to DSU in such a system is page reconstruction, as described by Kilpeläinen in \cite{Kilpelainen2023}. The method revolves around copying the page into a buffer in RAM, where changes can be applied, and then erasing and rewriting the page. This would add a significant memory usage overhead to the update process. In addition, if such a page required only a small change, the overhead of copying the entire page would be is wasteful. This approach would also introduce additional complexity to the update process, as the system would have to handle cases where update operations cross page boundaries. FRAM solves all these issues by allowing for bit-wise writes without the need for pre-erasing memory. FRAM is also a low power memory technology since it doesn't require a charge pump to write data, as opposed to Flash memory or EEPROM. This is particularly useful in battery-powered devices, where power consumption is a critical factor. In addition to the speed, power efficiency and non-volatility, FRAM also has a much higher endurance than Flash memory, with a specified $10^{15}$ write cycles, compared to to $10^{5}$ with flash memory \cite{framReport}. This makes it a suitable candidate for devices that are expected to run continuously for long periods of time, devices which would benefit greatly from the ability to update the software in the field.

One drawback of the FRAM implementation on the MSP430 platform is that accesses to it are limited to 8MHz. Thus, if the microcontroller is running at 16MHz, the CPU will be stalled for every access to FRAM \cite{framReport}. This is a limitation that must be considered when measuring the performance of the update process. The wait states require no user interaction apart from the initial setup of the microcontroller, and are handled automatically by the CPU.
