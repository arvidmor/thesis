\documentclass[]{report}
\usepackage[latin1]{inputenc}
\usepackage{times}
\usepackage{graphicx}
\begin{document}

 \newlength{\figurewidth}
\setlength{\figurewidth}{0.7\textwidth}

\newcommand{\codesize}{\scriptsize}

\title{Title}

\author{Author}

% check if front page needs something special lite stating
% what kind of thesis this is etc.

\maketitle
\tableofcontents
\listoffigures

% Make first version of abstract consist of four sentences. Fill in
% more later, if needed. 

\begin{abstract}
% First sentence: state the problem.

% Second sentence: describe why this is a problem.

% Third sentence: state your solution.

% Fourth sentence: sketch the consequences of your solution

\end{abstract}

% Make first version of the introduction section consist of four
% paragraphs. Add paragraphs later if needed.

\chapter{Introduction}
% usually 2-4 pages, you can use ``I'' or "We'' when you
% describe what you have done


% 1. Introduction to the problem and problem statement:
% why is the work needed/done, how will ``the world'' benefit from it.
% everybody with some education should be able to read this part
% and understand that the thesis is useful
% you can use sections for these parts\section{Motivation and Problem Statement}

% 2. how do you attack the problem including the method (analysis, 
% experimentation etc)

% 3. other possible approaches or solutions

% 4. describe your solution and the summarize the main results,
%    if possible the research contributions


% 5: Thesis structure: outline of rest of thesis


\chapter{Background}

% material necessary to understand the rest of the thesis

% last section in this chapter is sometimes related work

% note: none of your own work here. It must be very easy for the reader to
% understand what is your (new) work and what was there before.
% Note that readers who know the area, may skip this part

~\cite{dunkels04contiki}

\chapter{Design}
\label{sect:design}

\chapter{Implementation}
\label{sect:implementation}

% can be merged with Design

\chapter{Evaluation}
\label{sect:evaluation}

% often starts with the goal of the evaluation and the setup

% make sure that all graphs have units on the axes

% for each experiment:
% 1. Motivation (why should the reader care about this experiment)
% 2. setup (if different from the main setup)
% optional: expected result: "We expect that with a higher load the response
% time increases"
% 3. result: the graph shows that ....
% 4. anlysis: explain why the graph "shows what it shows". Explain non-obvious
% behaviours in the results (sometimes you need to make more/other measurements to
% support your analysis

\chapter{Conclusions and Future work}
\label{sect:conclusions}


\bibliography{ref}
\bibliographystyle{plain}

\end{document}
